\documentclass[../main/TOP_manual]{subfiles}

\begin{document}

\chapter{TOP structure and workflow}
\label{chap:ModDes}

TOP is the NEMO hardwired interface toward biogeochemical models, which provides the physical constraints/boundaries for oceanic tracers (Fig. ~\ref{fig:topstructure}).

Based on a modular structure, this component allows one to exploit available built-in modules and further develop a range of applications, spanning from the implementation of a dye passive tracer to evaluate dispersion processes (by means of MY\_TRC), track water masses age (AGE module), assess the ocean interior penetration of persistent chemical compounds (e.g., gases like CFC or even PCBs), up to the full set of equations to simulate marine biogeochemical cycles.

TOP interface has the following location in the code repository : \path{<nemo-repository>/src/TOP/}

and the following modules are available:

%-----------  tableau  ------------------------------------
\begin{itemize}
        \item \textbf{TRP}    	 :    Interface to NEMO physical core for computing tracers transport
        \item \textbf{CFC}	 :    Inert tracers (CFC11,CFC12, SF6)
        \item \textbf{C14}	 :    Radiocarbon passive tracer
        \item \textbf{AGE}	 :    Water age tracking
        \item \textbf{MY\_TRC}   :    Template for creation of new modules and external BGC models coupling
        \item \textbf{PISCES}    :    Built in BGC model. See \cite{aumont_2015} for a complete description
\end{itemize}
%----------------------------------------------------------

\begin{figure}[ht]
\begin{center}
\vspace{0cm}
\includegraphics[width=0.70\textwidth]{Fig_TOP_design}
\caption{Schematic view of the TOP interface within NEMO framework}
\label{fig:topstructure}
\end{center}
\end{figure}

\pagebreak

The workflow of the TOP interface within the NEMO framework is organized around two main blocks of the code, the initialization (trc\_ini) and time marching (trc\_stp) procedures, as in the following

\begin{minted}{bash}
nemogcm
    !
    nemo_ini            !   NEMO General Initialisations
         !
         trc_ini                               ! TOP  Initialisations
    !
    stp()                   !   NEMO Time-stepping
        !
        trc_stp()                            ! TOP time-stepping
            !
            trc_wri()           ! I/O manager : Output of passive tracers
            trc_sms()           ! Sinks and sources program manager
            trc_trp()            ! Transport of passive tracers
            trc_rst_wri()      ! Write tracer restart file
            trd_mxl_trc()     ! trends: Mixed-layer
\end{minted}

\subsection*{Model initialization (./src/TOP/trcini.F90)}

This module consists on inital set up of passive tracers variables and parameters  : read the namelist, set initial tracer fields (either read restart or read data or analytical formulation and  specific initailisation in each SMS module  ( analytical initialisation of tracers or constant values )

\begin{minted}{bash}
trc_init                              ! TOP  Initialisations
    !
    IF( PISCES )    trc_ini_pisces()     !  PISCES bio model
    IF( MY_TRC)    trc_ini_my_trc()    !  MY_TRC model
    IF( CFCs     )    trc_ini_cfc   ()       !  CFCs
    IF( C14       )    trc_ini_c14   ()       !  C14 model
    IF( AGE      )    trc_ini_age   ()       !  AGE tracer
    !
    IF( REST   )    trc_rst_read()         ! Restart from a file
    ELSE            trc_dta()                   ! Initialisation from data
\end{minted}

\subsection*{BGC trends computation (./src/TOP/trcsms.F90)}

This is the main module where the passive tracers source minus sinks of each TOP sub-module is managed.

\begin{minted}{bash}
trc_sms()                               ! Sinks and sources prooram manager
    !
    IF( PISCES  )    trc_sms_pisces()         ! main program of PISCES
    IF( CFCs     )    trc_sms_cfc()               ! surface fluxes of CFC
    IF( C14       )    trc_sms_c14()               ! surface fluxes of C14
    IF( AGE       )    trc_sms_age()              ! Age tracer
    IF( MY_TRC)    trc_sms_my_trc()         ! MY_TRC  tracers
\end{minted}

\subsection*{Physical trends computation (./src/TOP/TRP/trctrp.F90)}

This is the main module where the passive tracers transport is managed. All the physical trends is calculated ( advective \& diffusive trends, surface BC from freshwater or external inputs )

\begin{minted}{bash}
trc_trp()       ! Transport of passive tracers
    !
    trc_sbc()         ! Surface boundary condition of freshwater flux
    trc_bc()           ! Surface and lateral Boundary Conditions
    trc_ais()          ! Tracers from Antarctic Ice Sheet (icb, isf)
    trc_bbl()          ! Advective (and/or diffusive) bottom boundary layer scheme
    trc_dmp()        ! Internal damping trends
    trc_bdy()         ! BDY damping trends
    trc_adv()         ! Horizontal & Vertical advection
    trc_ldf()           ! Lateral mixing
    trc_zdf()          ! Vert. mixing & after tracer
    trc_atf()           ! Time filtering of "now" tracer fields
    trc_rad()         ! Correct artificial negative concentrations
\end{minted}

\subsection*{Outputs  (./src/TOP/TRP/trcwri.F90)}

This is the main module where the passive tracer outputs of each TOP sub-module is managed using the I/O library XIOS.

\begin{minted}{bash}
trc_wri()                               ! I/O manager : Output of passive tracers
!
IF( PISCES   )    trc_wri_pisces()      ! Output of PISCES diagnostics
IF( CFCs      )    trc_wri_cfc()            ! Output of Cfcs diagnostics
IF( C14         )    trc_wri_c14()           ! surface fluxes of C14
IF( AGE        )    trc_wri_age()           ! Age tracer
IF( MY_TRC )    trc_wri_my_trc()      ! MY_TRC  tracers
\end{minted}

\end{document}
