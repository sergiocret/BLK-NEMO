\documentclass[../main/TOP_manual]{subfiles}

\begin{document}

\chapter{ Model Setup}

The usage of TOP is activated i) by including in the configuration definition the component TOP and ii) by adding the macro key\_top in the configuration CPP file (see for more details “Learn more about the model”).
As an example, the user can refer to already available configurations in the code, ORCA2\_ICE\_PISCES being the NEMO biogeochemical demonstrator and GYRE\_BFM to see the required configuration elements to couple with an external biogeochemical model (see also Section 4).\\
Note that, since version 4.0, TOP interface core functionalities are activated by means of logical keys and all submodules preprocessing macros from previous versions were removed.\\

Below is the list of preprocessing keys that apply to the TOP interface (beside key\_top):
\begin{itemize}
	\item key\_xios use XIOS I/O
	\item key\_agrif enables AGRIF coupling
	\item key\_trdtrc and key\_trdmxl\_trc trend computation for tracers
\end{itemize}

There are only two entry points in the NEMOGCM model for passive tracers :
\begin{itemize}
	\item initialization (trcini) : general initialization of global variables and parameters of BGCM
	\item time-stepping (trcstp) : time-evolution of SMS first, then time evolution of tracers by transport
\end{itemize}

\section{ Setting up a passive tracer configuration}
%------------------------------------------namtrc_int----------------------------------------------------
\nlst{namtrc}
%-------------------------------------------------------------------------------------------------------------

As a reminder, the revisited structure of TOP interface now counts for five different modules handled in namelist\_top :

\begin{itemize}
        \item \textbf{PISCES}, default BGC model
        \item \textbf{MY\_TRC}, template for creation of new modules couplings (see section 3.2) or user defined passive tracer dynamics
        \item \textbf{CFC}, inert tracers dynamics (CFC$_{11}$,CFC$_{12}$,SF$_{6}$) updated based on OMIP-BGC guidelines (Orr et al, 2016)
        \item \textbf{C14}, radiocarbon passive tracer
        \item \textbf{AGE}, water age tracking
\end{itemize}

For inert, C14, and Age tracers, all variables settings (\textit{sn\_tracer} definitions) are hard-coded in \textit{trc\_nam\_*} routines. For instance, for Age tracer:
%------------------------------------------namtrc_int----------------------------------------------------
\nlst{nam_trc_age}
%---------------------------------------------------------------------------------------------------------

The modular approach was also implemented in the definition of the total number of passive tracers (jptra) which is specified by the user in  \textit{namtrc}

\section{ TOP Tracer Initialization}

Two main types of data structure are used within TOP interface to initialize tracer properties and to provide related initial and boundary conditions. 
In addition to providing name and metadata for tracers, the use of initial and boundary conditions is also defined here (\textit{sn\_tracer}).
The data structure is internally initialized by the code with dummy names and all initialization/forcing logical fields are set to \textit{false} .
Below are listed some features/options of the TOP interface accessible through the \textit{namelist\_top\_ref} and modifiable by means of \textit{namelist\_top\_cfg} (as for NEMO physical ones).

There are three options to initialize TOP tracers in the \textit{namelist\_top } file: (1) initialization to hard-coded constant values when \textit{ln\_trcdta} at \textit{false}, (2) initialization from files when \textit{ln\_trcdta} at \textit{true}, and (3) initialisation from restart files by setting \textit{ln\_rsttr} to \textit{true} in \textit{namelist}.

In the following, an example of the full structure definition is given for four tracers (DIC, Fe, NO$_{3}$, PHY) with initial conditions and different surface boundary and coastal forcings for DIC, Fe, and NO$_{3}$: 

%------------------------------------------namtrc_int----------------------------------------------------
\nlst{namtrc_cfg}
%---------------------------------------------------------------------------------------------------------

You have to activate which tracers (\textit{sn\_tracer}) you want to initialize by setting them to \texttt{true} in the  column. 

\nlst{namtrc_dta_cfg}

In \textit{namtrc\_dta}, you prescribe from which files the tracer are initialized (\textit{sn\_trcdta}). 
A multiplicative factor can also be set for each tracer (\textit{rn\_trfac}). 


\section{ TOP Boundaries Conditions}

\subsection{Surface and lateral boundaries}

Lateral and surface boundary conditions for passive tracers are prescribed in \textit{namtrc\_bc} as well as whether temporal interpolation of these files is enabled. Here we show the cases of Fe and NO$_{3}$ from dust and rivers with different output frequencies.
 
%------------------------------------------namtrc_bc----------------------------------------------------
\nlst{namtrc_bc_cfg}
%---------------------------------------------------------------------------------------------------------

\subsection{Antartic Ice Sheet tracer supply}

As a reminder, the supply of passive tracers from the AIS is currently implemented only for dissolved Fe. The activation of this Fe source is done by setting \textit{ln\_trcais} to \textit{true} and by adding the Fe tracer (\textit{sn\_tracer(2) = .true.}) in the 'ais' column in \textit{\&namtrc} (see section 2.2). \\

As the external source of Fe from the AIS is represented by associating  a sedimentary Fe content (with a solubility fraction) to the freshwater fluxes of icebergs and ice shelves, these fluxes have to be activated in \textit{namelist\_cfg}. The reading of the freshwater flux file from ice shelves is activated in \textit{namisf} with the namelist parameter \textit{ln\_isf} set to \textit{true}.

You have to choose between two options depending whether the cavities under ice shelves are open or not in your grid configuration:
\begin{itemize}
	\item ln\_isfcav\_mlt = .false. (resolved cavities)
	\item ln\_isfpar\_mlt = .true. (parameterized distribution for unopened cavities)
\end{itemize}

%------------------------------------------namisf----------------------------------------------------
\nlst{namisf_cfg_eORCA1}
%-----------------------------------------------------------------------------------------------------

Runoff from icebergs is activated by setting \textit{ln\_rnf\_icb} to \textit{true} in the \textit{\&namsbc\_rnf} section of \textit{namelist\_cfg}.

%------------------------------------------namsbc_rnf--------------------------------------------------
\nlst{namsbc_rnf_cfg_eORCA1}
%---------------------------------------------------------------------------------------------------------

The freshwater flux from ice shelves and icebergs is based on observations and modeled climatologies and is available for eORCA1 and eORCA025 grids :
\begin{itemize}
	\item runoff-icb\_DaiTrenberth\_Depoorter\_eORCA1\_JD.nc
	\item runoff-icb\_DaiTrenberth\_Depoorter\_eORCA025\_JD.nc 
\end{itemize}

%------------------------------------------namtrc_ais----------------------------------------------------
\nlst{namtrc_ais_cfg}
%---------------------------------------------------------------------------------------------------------

Two options for tracer concentrations in iceberg and ice shelf can be set with the namelist parameter \textit{nn\_ais\_tr}:
\begin{itemize}
	\item 0 : null concentrations corresponding to dilution of BGC tracers due to freshwater fluxes from icebergs and ice shelves
	\item 1 : prescribed concentrations set with the \textit{rn\_trafac} factor
\end{itemize}

The depth until which Fe from melting iceberg is delivered can be set with the namelist parameter \textit{rn\_icbdep}. The value of 120 m is the average underwater depth of the different iceberg size classes modeled by the NEMO iceberg module, which was used to produce the freshwater flux climatology of icebergs.


\end{document}
